\section{Analyse der Arbeitserleichterung}\label{Analyse der
Arbeitserleichterung}
Im vorherigen Kapitel wurden 4 Arbeitschritte ausfindig gemacht, die mittels
\emph{Xtext} direkt effizienter gestaltet werden k�nnen. 
\begin{enumerate}
\item Pr�fungskonfiguration definieren (siehe Kapitel \ref{TP2})
\item Codereview (siehe Kapitel \ref{TP2})
\item Fehlersuche (des Kunden, siehe Kapitel \ref{TP3})
\item Fehlersuche (der deg, siehe Kapitel \ref{TP3})
\end{enumerate}
\subsection{Effizienzsteigerung mit \emph{Xtext} durch eine Grammatik}
F�r die folgende Analyse wird in Betracht gezogen, dass f�r die Sprache zur
Definition von Prf�ngskonfigurationen eine Grammatik existiert, die von
\emph{Xtext} gelesen werden kann. Diese Grammatik wurde im Praxisbericht
\emph{Entwicklung einer Grammatik f�r eine DSL mit xText am Beispiel einer
Sprache zur Definition von Pflichtpr�fungen in profil c/s}\cite{pb5}
entwickelt. Um alle Konfigurationsdateien bzgl. der Syntax zu validieren, waren
noch einige Erweiterungen der Grammatik von N�ten (siehe Anhang
\ref{Grammatik}).
Beispielsweise mussten Kommentare in die Menge der Terminale aufgenommen werden.
\begin{lstlisting}[caption = Kommentare mit '\#' als Terminale]
terminal SL_COMMENT:
	'#' !('\n' | '\r')* ('\r'? '\n')?;
\end{lstlisting}
Dar�berhinaus wurden die Terminale zur Definition von \emph{Aktionen},
\emph{Wirkungen} und \emph{Klassennamen} auf die in profil c/s verwendeten
Bezeichnungen eingeschr�nkt. Listing \ref{wirkungen} zeigt dies f�r die
\emph{Wirkungen}. Die Umsetzung f�r die \emph{Aktionen} und \emph{Klassennamen}
ist analog dazu im Anhang \ref{Grammatik} zu finden.
\begin{lstlisting}[caption = Terminale f�r Wirkungen]
WIRKUNG:
	'VERHINDERT_AKTION' | 'OHNE' | 'WARNUNG';
\end{lstlisting}\label{wirkungen}
Erzeugt man mit \emph{Xtext} einen Editor, der die verwendete Syntax in den
Konfigurationsdateien hinsichtlich dieser Grammatik �berpr�ft, k�nnen dadurch
folgende Fehlerquellen ausgeschlossen werden.
\begin{itemize}
  \item Falscher struktureller Aufbau
  \item Zuweisung von Aktionen, die in profil c/s nicht existieren
  \item Zuweisung von Wirkungen, die in profil c/s nicht existieren
  \item Zuweisung von Klassennamen (Pr�fungsalgorithmen), die in profil c/s
  nicht existieren
\end{itemize}
Voraussetzung daf�r ist jedoch, dass \emph{Aktionen}, \emph{Wirkungen} und
\emph{Klassennamen} auch in der Grammatik gepflegt werden.\\
Die Arbeit w�re durch die wegfallenden Fehlerquellen effizienter. Dar�ber hinaus
muss der Entwickler auch nicht mehr nach dem vollqualifizierten Klassennamen
suchen, weil ihn dieser vom Editor vorgschlagen wird. Genauso verh�lt sich das
mit der Suche nach den richtigen Bezeichnungen f�r die \emph{Aktionen} oder
\emph{Wirkungen}\\
Beim Codereview muss dabei bedacht werden, dass damit noch keine semantik
validiert wird. Besnoders ist darauf zu achten, dass s�mtliche IDs korrekt
deklariert und zugewiesen sind.\\
Bei der Fehlersuche kann somit ein syntaktischer Fehler ausgeschlossen
werden.\footnote{Sollte es dennoch zu einem Syntaxfehler kommen, muss die
Grammatik angepasst werden.} Die Mitarbeiter der deg k�nnen sich bei der
Fehlersuche auf die fachliche Korrektheit der Pr�fungskonfiguration
konzentrieren. Der Kunde geniest bei der Fehlersuche dadurch derzeit kein
Vorteil. W�re der Kunde jedoch in der Lage die Konfigurationsdateien zu lesen
und zu verstehen, k�nnte auch dieser die Fachlichkeit der Pr�fungskonfiguration
pr�fen und sie mit dem Angebot abgleichen.\footnote{Neben einem Fehler in der 
Konfiguration kann auch ein vom Kunden nicht ausreichend gepr�ftes,
aber angenommenes Angebot zu einem nicht
erwarteten Ergebnis des Tests f�hren.}


\subsection{Effizienzsteigerung mit \emph{Xtext} durch Validierung}
Mittels Validierungen k�nnen bei \emph{Xtext} semantische Zusammenh�nge gepr�ft
werden. In dem Beispiel der Konfigurationsdateien werden alle Eigenschaften der
Pr�fungskonfigurationen �ber IDs zugewiesen. Die IDs, zu denen die Eigenschaften
zugewiesen werden, m�ssen in der Datei ein mal deklariert worden sein. Anderen
Falls ist die Zuweisung einer Eigenschaft �ber diese ID �berfl�ssig.\\
Ein weiterer semantischer Aspekt ist die Laufende Nummer, die bei der Zuweisung
mehrerer \emph{Wirkungen} und \emph{Aktionen} benutzt wird.\footnote{Genaueres
dazu ist im Praxisbericht \emph{Entwicklung einer Grammatik f�r eine DSL mit
xText am Beispiel einer Sprache zur  Definition von Pflichtpr�fungen in profil c/s}\cite{pb5} zu finden.}
