\section{Grundlegende Begriffe}
\subsection*{Gesch�ftsprozess}
Unter einem Gesch�ftsprozess (Prozess) versteht man eine Zusammenfassung der
erforderlichen betrieblichen Abl�ufe, die zur Erstellung
von Produkten und Dienstleistungen notwendig sind.\cite[S. 4]{gpm}
Bei der Darstellung mittels der EPK\footnote{Ereignisgesteuerte Prozesskette}
(welche in diesem Bericht verwendet wird) werden Prozesse mittels Zust�nden und
Funktionen visualisiert.

\subsection*{Workflow}
Ein Workflow hingegen beschreibt dar�berhinaus auch die beteiligten Personen und
die von ihnen auszuf�hrenden Arbeitsschritte innerhalb des
Prozesses.(\cite{workflow}) Es handelt sich um eine genaue Beschreibung
der Arbeitsschritte, die abgearbeitet werden m�ssen, um den Prozess von Anfang bis
Ende auszuf�hren.

\subsection*{Syntax}


\subsection*{Semantik}

\subsection*{Grammatik} 
Da in Kaptiel
\ref{Analyse der Arbeitserleichterung} auf Grammatiken f�r \emph{Xtext} Bezug
genommen wird, ist es wichtig hier zu definieren, was unter einer Grammatik verstanden wird. 
Eine Grammatik definiert die Syntax einer Sprache und gibt somit vor, welche 
Sprachkonstrukte bei der Verwendung der Sprache benutzt werden d�rfen. \cite[Seite 26]{theoinfo}

\subsection*{Xtext}
\emph{Xtext} ist ein Framework mit dem es zum eine m�glich ist, in sehr kurzer
Zeit neue Sprachen zu entwickeln. Es bietet aber auch f�r bestehende Sprachen in
Verbindung mit Eclipse ein entsprechendes Tooling an, mit dem sich die Arbeit
mit dieser Sprache effizienter gestalten l�sst. Das �u�ert sich darin, dass
mit \emph{Xtext} Editoren generiert werden k�nnen, die Fehler in der Sprache
erkennen und den Nutzer darauf hinweisen, wie dieser es bei der Nutzung von
anderen Sprachen in der Eclipse IDE gewohnt ist.\cite[S. 113]{xtextdoc}
