\section{Grundlegende Begriffe}
In Kapitel \ref{GP} wird Bezug auf die Begriffe
\emph{Gesch�ftsprozess (Prozess)} und \emph{Workflow} genommen. Aus diesem Grund
wird hier eine Abgrenzung zwischen den beiden Begriffen vorgenommen. Weiterhin
werden in Kapitel \ref{Analyse der Arbeitserleichterung} die Begriffe
\emph{Syntax} und \emph{Semantik} verwendet. Wie diese Begriffe in diesem
Bericht zu verstehen sind, ist ebenfalls in diesem Kapitel definiert.

\subsection*{Gesch�ftsprozess}
Unter einem Gesch�ftsprozess (Prozess) versteht man eine Zusammenfassung der
erforderlichen betrieblichen Abl�ufe, die zur Erstellung
von Produkten und Dienstleistungen notwendig sind.\vcite{gpm}{4}
Bei der Darstellung mittels der EPK\footnote{Ereignisgesteuerte Prozesskette}
(welche in diesem Bericht verwendet wird) werden Prozesse durch Zust�nde und
Funktionen visualisiert.

\subsection*{Workflow}
Ein Workflow beschreibt die beteiligten Personen und
die von ihnen auszuf�hrenden Arbeitsschritte innerhalb des
Prozesses.\cite{workflow} Es handelt sich um eine genaue Beschreibung
der Arbeitsschritte, die abgearbeitet werden m�ssen, um den Prozess von Anfang bis
Ende durchzuf�hren.

\subsection*{Syntax}
Die Syntax ist ein Bestandteil einer Sprache. Durch die Syntax wird die Notation
der Sprache definiert. Sie beschreibt demnach die Sprachkonstrukte die f�r eine
Sprache zul�ssig sind und mit denen man sich in dieser Sprache ausdr�cken kann.
In Bezug auf Programmiersprachen beschreibt die Syntax somit die
Sprachkonstrukte, mit denen der Nutzer ein Programm beschreiben
kann.\vcite{voelter}{26} Zur Beschreibung einer Syntax wird eine
\emph{Grammatik} verwendet\footnote{Beschreibung einer Grammatik ist im
Praxisbericht \emph{Entwicklung einer
Grammatik f�r eine DSL mit xText am Beispiel einer Sprache zur Definition von 
Pflichtpr�fungen in profil c/s}\cite{pb5} zu finden.}.\vcite{theoinfo}{26}

\subsection*{Semantik}
Die Semantik ist ebenfalls ein Bestandteil einer Sprache. Sie enth�lt die Regeln
f�r die strukturelle Zusammensetzung der Sprachkonstrukte, die von der Syntax
definiert werden. Bezogen auf Programmiersprachen beschreibt die Semantik die
Bedeutung aufeinander folgender Sprachkonstrukte. \vcite{voelter}{26}

\subsection*{Xtext}
\emph{Xtext} ist ein Framework mit dem es einerseits m�glich ist, in sehr kurzer
Zeit neue Sprachen zu entwickeln. Andererseits bietet es auch f�r
bestehende Sprachen in Verbindung mit Eclipse ein entsprechendes Tooling an, mit dem sich die Arbeit
mit dieser Sprache effizienter gestalten l�sst. Das �u�ert sich darin, dass
mit \emph{Xtext} Editoren generiert werden k�nnen die Fehler in der Sprache
erkennen und den Nutzer darauf hinweisen. Diese Fehler werden dabei wie gewohnt
in der Eclipse IDE angezeigt.\vcite{xtextdoc}{113}
