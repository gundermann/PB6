\section{Fazit}
In diesem Bericht wurde gezeigt, dass durch den Einsatz von \emph{Xtext} einige
Fehlerquellen bei der Definition von Pr�fungskonfigurationen ausgeschlossen
werden k�nnen.
Voraussetzung daf�r ist jedoch eine gut definierte Grammatik, um eine  korrekte Syntax zuzusicher. Weiterhin bedarf
es f�r die Absicherung semantischer Zusammenh�nge entsprechende
Validierungsregeln. Mit \emph{Xtext} l�sst sich beides umsetzen und sehr
einfach in Eclipse intergrieren.\\
Durch den Ausschluss bestimmter Fehlerquellen l�sst sich bei den
Arbeitschritten, zu deren Unterst�tzung \emph{Xtext} eingesetzt werden kann
(genannt zu Beginn von Kaptiel \ref{Analyse der Arbeitserleichterung}), eine
Effizientsteigerung erzielen. Das hat auch eine Auswirkung auf den Gesamtprozess
(siehe Kapitel \ref{GP}). Auch das Durchf�hren von Prozessschleifen,
die Aufgrund von fehlgeschlagener Tests auftreten, wird im
Gesamtprozess reduziert.
Demnach wird der Prozess beim Einsatz von \emph{Xtext} schneller einen Endzustand erreichen.\\
Als zus�tzlichen Arbeitsaufwand k�nnte man die Pflege der Grammatik betrachten.
Diese muss immer aktualisiert werden, wenn \emph{Akionen} oder
\emph{Klassennamen} neu hinzukommen, oder wenn an den Bezeichnungen
der bestehenden \emph{Akionen} und
\emph{Klassennamen} etwas ge�ndert wird. Das Editieren der Grammatik ist jedoch
auch in Bezug auf diesen Aspekt ein Vorteil. Wenn die Grammatik ge�ndert wird,
werden in den Konfigurationsdateien an den entsprechenden Stellen Fehler
angezeigt. Dadurch entf�llt das Suchen entsprechender Stellen in den
Konfigurationsdateien, was bei vielen Konfigurationen ein erheblich gr��erer
Arbeitsaufwand w�re, als das �ndern einer Zeile in der Grammatik.
