\section{Fazit}
Damit wurde gezeigt, dass durch den Einsatz von \emph{Xtext} einige
Fehlerquellen ausgeschlossen werden k�nnen. Voraussetzung daf�r ist jedoch eine
gut definierte Grammatik, um eine  korrekte Syntax zuzusicher. Weiterhin bedarf
es f�r die Absicherung semantische Zusammenh�nge entsprechende
Validierungsregeln. Mit \emph{Xtext} l�sst sich beides umsetzen und sehr
einfach in Eclipse intergrieren.\\
Durch den Ausschluss bestimmter Fehlerquellen wird bei den Arbeitschen, zu deren
Unterst�tzung \emph{Xtext} eingesetzt werden kann (genannt zu Beginn von Kaptiel
\ref{Analyse der Arbeitserleichterung}), eine Effizientsteigerung erzielt. Das
hat auch eine Auswirkung auf den Gesamtprozess (siehe Kapitel \ref{GP}). Bei
diesem wird das Durchf�hren von ProzessSchleifen, die Aufgrund von Fehler
auftreten, reduziert. Demnach wird der Prozess beim Einsatz von \emph{Xtext}
schneller einen Endzustand erreichen.\\
Als zus�tzlichen Arbeitsaufwand k�nnte man die Pflege der Grammatik betrachten.
Diese muss immer aktualisiert werden, wenn \emph{Akionen} oder
\emph{Klassennamen} neu hinzukommen, oder wenn an den Bezeichnungen
der besteheneden etwas ge�ndert wird. Das Editieren der Grammatik ist jedoch
auch in Bezug auf diesen Aspekt ein Vorteil. Wenn die Grammatik ge�ndert wird,
werden in den Konfigurationsdateien an den entsprechenden Stellen Fehler
angezeigt. Somit entf�llt das Suchen entsprechender Stellen in den
Konfigurationsdateien, was bei vielen Konfigurationen ein erheblich gr��erer
Arbeitsaufwand w�re.\\
Allgemein steht aus meiner Sicht dem Einsatz von \emph{Xtext} in der deg f�r
die Definition von Pr�fungskonfigurationen nicht mehr im Weg.
