
\usepackage[german]{babel}
\usepackage[ansinew]{inputenc}
\usepackage[T1]{fontenc} %Umlaute
\usepackage{lmodern}
\usepackage[top=2cm, left=4cm, bottom=2cm, right=3cm]{geometry}
\usepackage{graphicx}
\usepackage{listings} 
\usepackage{multirow}
\usepackage{color}		 % f�r Farben im allgemeinen
\usepackage{colortbl}
\usepackage{cite}
\usepackage{bibgerm}
\usepackage{palatino}
\usepackage{chngpage}
\usepackage{rotating}
\usepackage{pdflscape}
\usepackage{float}
\renewcommand{\floatpagefraction}{0.85}

\usepackage{fancyhdr}
\pagestyle{fancy}



\fancyhf{}
\fancyhead[RE]{\slshape \nouppercase{\leftmark}}    % Even page header: "page   chapter"
\fancyhead[LO]{\slshape \nouppercase{\rightmark}}   % Odd  page header: "section   page"
\fancyhead[RO,LE]{\bfseries \thepage} 
\renewcommand{\headrulewidth}{1pt}    % Underline headers
\renewcommand{\footrulewidth}{0pt}    

\fancypagestyle{plain}{               % No chapter+section on chapter start pages
\fancyhf{}
\fancyhead[RO,LE]{\bfseries \thepage}
\renewcommand{\headrulewidth}{1pt}
\renewcommand{\footrulewidth}{0pt}
}

% Left headings: "1  INTRODUCTION"
%\renewcommand{\chaptermark}[1]{%
%\markboth{\thechapter\ \ \ \ #1}{}}

% Right headings: "1.1  Basics"
\renewcommand{\sectionmark}[1]{%
\markright{\thesection\ \ \ \ #1}{}}

\lstset{language=java} 
\lstset{basicstyle=\scriptsize}
\lstset{numbers=left, numberstyle=\tiny, numbersep=2pt, breaklines=true} 
\graphicspath{{Bilder/}}
\newcommand{\listoflolentryname}{\lstlistingname} 
%\newcommand{\fullcite}{\citep} %for "Author [1980]"
\usepackage[pdftex,plainpages=false,pdfpagelabels]{hyperref}
% --- Farbdefinitionen ----------------------------------------
\definecolor{rot}{rgb}{1,0.3,0}
\definecolor{gelb}{rgb}{1,1,0}
\definecolor{gruen}{rgb}{0,1,0.4}
\definecolor{darkblue}{rgb}{0.2,0.3,1}
\definecolor{lightblue}{rgb}{0.6,0.7,1}
\definecolor{white}{rgb}{1,1,1}

\linespread{1.5}

\bibliographystyle{geralpha}

%----------------------------------------------------------------------------------
% \vcite	
%				{ source }
%				{ page }
%
%Direktes Zitat
\newcommand{\simplevcite}[1]
{
(vgl. \cite{#1})
}

%----------------------------------------------------------------------------------
% \vcite	
%				{ source }
%				{ page }
%
%Direktes Zitat
\newcommand{\vcite}[2]
{
(vgl. \cite[S.#2]{#1})
}

\newcommand{\myHugeFigure}[3]
{
\begin{sidewaysfigure}[H]
	
		\includegraphics[width= \textheight]{#1}
		\caption{#2}
		\label{#3}
	
\end{sidewaysfigure}
}

